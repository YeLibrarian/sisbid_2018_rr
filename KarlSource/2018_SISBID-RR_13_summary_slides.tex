\documentclass[12pt,t]{beamer}
\usepackage{graphicx}
\setbeameroption{hide notes}
\setbeamertemplate{note page}[plain]
\usepackage{listings}

% get rid of junk
\usetheme{default}
\beamertemplatenavigationsymbolsempty
\hypersetup{pdfpagemode=UseNone} % don't show bookmarks on initial view


% font
\usepackage{fontspec}
\setsansfont
  [ ExternalLocation = fonts/ ,
    UprightFont = *-regular ,
    BoldFont = *-bold ,
    ItalicFont = *-italic ,
    BoldItalicFont = *-bolditalic ]{texgyreheros}
\setbeamerfont{note page}{family*=pplx,size=\footnotesize} % Palatino for notes
% "TeX Gyre Heros can be used as a replacement for Helvetica"
% I've placed them in fonts/; alternatively you can install them
% permanently on your system as follows:
%     Download http://www.gust.org.pl/projects/e-foundry/tex-gyre/heros/qhv2.004otf.zip
%     In Unix, unzip it into ~/.fonts
%     In Mac, unzip it, double-click the .otf files, and install using "FontBook"

% named colors
\definecolor{offwhite}{RGB}{255,250,240}
\definecolor{gray}{RGB}{155,155,155}

\ifx\notescolors\undefined % slides
  \definecolor{foreground}{RGB}{255,255,255}
  \definecolor{background}{RGB}{24,24,24}
  \definecolor{title}{RGB}{107,174,214}
  \definecolor{subtitle}{RGB}{102,255,204}
  \definecolor{hilit}{RGB}{102,255,204}
  \definecolor{vhilit}{RGB}{255,111,207}
  \definecolor{lolit}{RGB}{155,155,155}
  \definecolor{codehilit}{RGB}{255,111,207}
\else % notes
  \definecolor{background}{RGB}{255,255,255}
  \definecolor{foreground}{RGB}{24,24,24}
  \definecolor{title}{RGB}{27,94,134}
  \definecolor{subtitle}{RGB}{22,175,124}
  \definecolor{hilit}{RGB}{122,0,128}
  \definecolor{vhilit}{RGB}{255,0,128}
  \definecolor{lolit}{RGB}{95,95,95}
  \definecolor{codehilit}{RGB}{24,24,24}
\fi
\definecolor{nhilit}{RGB}{128,0,128}  % hilit color in notes
\definecolor{nvhilit}{RGB}{255,0,128} % vhilit for notes

\newcommand{\hilit}{\color{hilit}}
\newcommand{\vhilit}{\color{vhilit}}
\newcommand{\nhilit}{\color{nhilit}}
\newcommand{\nvhilit}{\color{nvhilit}}
\newcommand{\lolit}{\color{lolit}}

% use those colors
\setbeamercolor{titlelike}{fg=title}
\setbeamercolor{subtitle}{fg=subtitle}
\setbeamercolor{institute}{fg=lolit}
\setbeamercolor{normal text}{fg=foreground,bg=background}
\setbeamercolor{item}{fg=foreground} % color of bullets
\setbeamercolor{subitem}{fg=lolit}
\setbeamercolor{itemize/enumerate subbody}{fg=lolit}
\setbeamertemplate{itemize subitem}{{\textendash}}
\setbeamerfont{itemize/enumerate subbody}{size=\footnotesize}
\setbeamerfont{itemize/enumerate subitem}{size=\footnotesize}

% page number
\setbeamertemplate{footline}{%
    \raisebox{5pt}{\makebox[\paperwidth]{\hfill\makebox[20pt]{\lolit
          \scriptsize\insertframenumber}}}\hspace*{5pt}}

% add a bit of space at the top of the notes page
\addtobeamertemplate{note page}{\setlength{\parskip}{12pt}}

% default link color
\hypersetup{colorlinks, urlcolor={hilit}}

\ifx\notescolors\undefined % slides
  % set up listing environment
  \lstset{language=bash,
          basicstyle=\ttfamily\scriptsize,
          frame=single,
          commentstyle=,
          backgroundcolor=\color{darkgray},
          showspaces=false,
          showstringspaces=false
          }
\else % notes
  \lstset{language=bash,
          basicstyle=\ttfamily\scriptsize,
          frame=single,
          commentstyle=,
          backgroundcolor=\color{offwhite},
          showspaces=false,
          showstringspaces=false
          }
\fi

% a few macros
\newcommand{\bi}{\begin{itemize}}
\newcommand{\bbi}{\vspace{24pt} \begin{itemize} \itemsep8pt}
\newcommand{\ei}{\end{itemize}}
\newcommand{\ig}{\includegraphics}
\newcommand{\subt}[1]{{\footnotesize \color{subtitle} {#1}}}
\newcommand{\ttsm}{\tt \small}
\newcommand{\ttfn}{\tt \footnotesize}
\newcommand{\figh}[2]{\centerline{\includegraphics[height=#2\textheight]{#1}}}
\newcommand{\figw}[2]{\centerline{\includegraphics[width=#2\textwidth]{#1}}}


%%%%%%%%%%%%%%%%%%%%%%%%%%%%%%%%%%%%%%%%%%%%%%%%%%%%%%%%%%%%%%%%%%%%%%
% end of header
%%%%%%%%%%%%%%%%%%%%%%%%%%%%%%%%%%%%%%%%%%%%%%%%%%%%%%%%%%%%%%%%%%%%%%

% title info
\title{13: Summary + Extras}
\author{}
\date{\href{https://bit.ly/2018rr}{\tt \color{foreground} bit.ly/2018rr}}

\begin{document}

% title slide
{
\setbeamertemplate{footline}{} % no page number here
\frame{
  \titlepage
  \note{We'll briefly summarize all that we discussed, and further
    touch on how to share your research data and code. Then we'll give
    pointers to the many things that we didn't have time to talk about.
} } }


\begin{frame}[c]{}

\begin{center}
\large
The most important tool is the {\hilit mindset},\\
when starting, that the end product \\
will be reproducible.
\end{center}

\hfill
{\lolit
{\textendash} \href{http://odin.mdacc.tmc.edu/~kabaggerly/}{Keith Baggerly}
}

\note{So true. Desire for reproducibility is step one.
}
\end{frame}




\begin{frame}[c]{Steps toward reproducible research}


  \bi
\item Slow down
\item Organize; document
\item Everything with code
\item Scripts → RMarkdown
\item Code → functions → packages
\item Version control with Git
\item Automation with Make
\item Choose a license
\item Share your work with others
  \ei


\note{Moving from ``standard practice'' to ``fully reproducible'' is
  hard. There are a lot of tools to learn and a lot of workflow
  changes to make. Don't try to change everything all at once. Focus
  on improving one aspect at a time, ideally jointly with your friends
  and colleagues. Your goal should be to have each project be a bit
  better organized than the previous.
}
\end{frame}



\begin{frame}[c]{Organize your data}

\only<1 | handout 0>{\figw{Figs/excel_bad_1.png}{0.95}}
\only<2 | handout 0>{\figw{Figs/excel_bad_2.png}{0.95}}
\only<3 | handout 0>{\figw{Figs/excel_good.png}{0.95}}
\only<4>{
  \begin{center}
    \href{https://kbroman.org/dataorg}{\Large \tt kbroman.org/dataorg}

\bigskip \bigskip

    Broman \& Woo (2018) Data organization \\ in spreadsheets. Am Stat 78:2--10

    \href{https://doi.org/10.1080/00031305.2017.1375989}{\tt
      \footnotesize doi:10.1080/00031305.2017.1375989}
   \end{center}
}


\note{Code should be human readable; data should be computer-readable.

      Suggestions at my web site; now a proper article in \emph{The American Statistician}.}
\end{frame}


\begin{frame}[c]{Challenges}


  \bi
\item Daily maintenance
  \bi
\item READMEs up to date?
\item Documentation matches code?
  \ei
\item Cleaning up the junk
  \bi
\item Move defunct stuff into an {\tt Old/} subdirectory?
  \ei
\item Start over from the beginning, nicely?
  \ei


\note{The organization of a project is dependent on your worst day
  with it. You keep everything carefully arranged for months and then
  one day you need to rush, rush, rush to get a manuscript out the
  door, and you leave a big mess.

  And a common problem is that you don't really know what you're doing
  until it's all done. So if you can, spend a week at the end making a
  separate, clean version of the whole thing.
}
\end{frame}



\begin{frame}{Sharing your work}

\vspace{12mm}

  \bi
\item Why share?
  \bi
\item Funding agency or journal requirement
\item Increased visibility
\item So that others can build on your work
  \ei
\item When?
  \bi
\item Continuously and instantaneously
\item When you submit a paper
\item When your paper appears
  \ei
\item Risks?
  \ei

\vspace{12mm}

\hfill \href{https://bit.ly/rr_sharing_slides}{\tt \footnotesize \lolit bit.ly/rr\_sharing\_slides}

  \note{
    There a lot of advantages to making your work public, and you may
    be required to make it public.

    Some scientists put all of the work in the open as they're doing
    it. Others wait until they submit a manuscript; others until the
    manuscript actually appears. Still others want to delay releasing
    data yet further. The earlier the better, I think.

    Are there risks? Many worry about being scooped: having someone
    find something cool in your data before you got a chance to find
    it yourself. I think the risk of this is far outweighed by the
    advantages of having data and code in the open. You may also worry
    about people finding problems in your analyses. But if there are
    problems, wouldn't you rather know about them? You can't build
    upon it if it's not right.

    The biggest risk is that you'll be ignored. Sharing more,
    sooner, and in a more convenient form will encourage broader
    visibility.
  }


\end{frame}


\begin{frame}[c]{}

\begin{center}
  I'm not worried about being scooped, \\
  I'm worried about being ignored.
\end{center}

\hfill
{\lolit
{\textendash} \href{https://twitter.com/BaxterTwi/status/887280515593756672}{Magnus Nordborg}
}

\note{I'm not entirely sure that this is the original source of the idea.
}
\end{frame}


\begin{frame}[c]{}

  \bbi
\item Share more
\item Share sooner
\item Share in a way that makes it easy for others to learn from and
  build upon
  \ei

  \note{
    It's easy for me to say, but I think the benefits, to the
    community, of open science far outweigh the risks to individual
    investigators. Sort of like vaccines.
    }

\end{frame}



\begin{frame}[c]{What to share?}

  \bi
\item For sure
  \bi
\item Primary dataset
\item Metadata
\item Data cleaning scripts
\item Analysis scripts
  \ei
\item It could help
  \bi
\item Very-raw data
\item Processed/clean data
\item Intermediate results
  \ei
\item No
  \bi
\item Confidential data (e.g. HIPAA data)
\item Passwords, private keys
  \ei
  \ei

  \note{It can be tricky to define what is the ``primary dataset''.
    How raw of data should you share? Where would someone want to
    start, and how can you make it easiest for them to dig into your
    analyses or make use of the data for other purposes, such as for a
    meta-analysis study? In some cases, it can be valuable to share
    many of the intermediate results, so that folks can jump into the
    bit that they care about without having to first run a bunch of
    complex and time-consuming analyses to get there.

    You definitely want to make sure that you're not including any
    confidential information, particularly regarding patient-level
    data, but also private keys for data APIs which you might be
    including in your scripts.
}
\end{frame}



\begin{frame}[c]{Where to share?}

  \bi
\item Domain-specific repository
  \bi
  \item Genbank, dbGaP, etc.
  \item See \href{http://re3data.org}{\tt re3data.org}
  \ei
\item \href{https://figshare.com}{Figshare}, \href{https://datadryad.org}{Dryad}, \href{https://zenodo.org}{Zenodo}
\item Institutional repository
  \item \href{https://github.com}{GitHub}, \href{https://bitbucket.org}{BitBucket}
  \ei

  \note{Share code at GitHub or BitBucket. But how about data?
    Consider domain-specific repositories like dbGaP, then general
    data repositories like Zenodo, and then look to see whether your
    institution has a data repository.

    Another option is as supplemental material for a publication, at
    the Journal's website, but this is often the most cumbersome for
    users.}
\end{frame}


\begin{frame}[c]{Resources}


\centerline{  \href{https://bit.ly/2018rr_resources}{\Large \tt bit.ly/2018rr\_resources} }



  \note{ Links to some resources related to the course material. }
\end{frame}


\begin{frame}<handout:0>[c]{}

  \begin{center}
    \large
    Some of the things we didn't cover
  \end{center}
\end{frame}



\begin{frame}[c]{Coding conventions}

  \bbi
\item[] Why are they cool?
  \bi
  \item They help you keep things consistent between team members
  \item They make code easier to read, and more likely to be used
  \ei
\item[] Why didn't we cover them?
    \bi
    \item Not enough time
    \ei
\item[] Where would we point you?
    \bi
    \item Hadley's recommendations
  \href{http://adv-r.had.co.nz/Style.html}{\tt \footnotesize
    adv-r.had.co.nz/Style.html}
    \item Google's recommendations
  \href{https://google.github.io/styleguide/Rguide.xml}{\tt
      \footnotesize google.github.io/styleguide/Rguide.xml}
\item Tidyverse style guide
  \href{http://style.tidyverse.org/}{\tt \footnotesize
    style.tidyverse.org}
   \ei
\ei

 \note{
   Coding conventions help to keep things consistent and make code
   easier to read.
 }

\end{frame}






\begin{frame}[c]{Code review}

  \bbi
\item[] Why is it cool?
  \bi
  \item Helps to find bugs and clean up confusing bits
  \item Potentially a test of the reproducibility of your work
  \ei
\item[] Why didn't we cover it?
    \bi
    \item Not enough time
    \ei
\item[] Where would we point you?
    \bi
  \item Software Carpentry blog post,
    \href{http://bit.ly/swc_codereview}{\tt bit.ly/swc\_codereview}
  \item Titus Brown's blog post,
    \href{http://bit.ly/titus_codereview}{\tt http://bit.ly/titus\_codereview}
   \ei
\ei

 \note{
   Working together with another person and reviewing each other's
   code can lead to better code and better projects.
 }

\end{frame}




\begin{frame}[c]{Software testing}

  \bbi
\item[] Why is it cool?
  \bi
  \item Explicit tests help you to avoid bugs, and to find bugs sooner
  \ei
\item[] Why didn't we cover it?
    \bi
    \item Not enough time
    \ei
\item[] Where would we point you?
    \bi
  \item testthat package,
    \href{https://github.com/hadley/testthat}{\tt github.com/hadley/testthat}
  \item
    \href{https://www.amazon.com/Testing-Code-Chapman-Hall-CRC/dp/1498763650}{Testing
      R Code} book
   \ei
\ei

 \note{
   Hadley wrote, ``It's not that we don't test our code, it's that we don't store our tests so they can be re-run automatically.''

   Including explicit tests and running them repeatedly will help you
   to avoid bugs or at least to find bugs sooner.

 }

\end{frame}

\begin{frame}[c]{Continuous integration (eg Travis)}

  \bbi
\item[] Why is it cool?
  \bi
  \item Automatically build and run tests when you push to GitHub
  \item Pull requests are automatically tested
  \ei
\item[] Why didn't we cover it?
    \bi
    \item Not enough time
    \ei
\item[] Where would we point you?
    \bi
  \item Julia Silge blog post,
    \href{https://juliasilge.com/blog/beginners-guide-to-travis/}{\tt juliasilge.com/blog/beginners-guide-to-travis}
  \item Hadley's R packages book,
    \href{http://r-pkgs.had.co.nz/check.html\#travis}{\tt r-pkgs.had.co.nz/check.html\#travis}
   \ei
\ei

 \note{
   Travis makes it easy to run packages tests regularly.
 }

\end{frame}







\begin{frame}[c]{Capturing dependencies}

  \bbi
\item[] Why is it cool?
  \bi
  \item Ensure that your carefully constructed reproducible project
    doesn't fail due to a change in one of the packages you use
  \ei
\item[] Why didn't we cover it?
    \bi
    \item Not enough time
    \ei
\item[] Where would we point you?
    \bi
  \item packrat package, \href{https://github.com/rstudio/packrat}{\tt github.com/rstudio/packrat}
  \item checkpoint package,
    \href{https://github.com/RevolutionAnalytics/checkpoint}{\tt github.com/RevolutionAnalytics/checkpoint}
   \ei
\ei

 \note{
   Saving copies of the exact versions of packages that you use can
   help ensure that your reproducible project will continue to be
   reproducible.
 }

\end{frame}




\begin{frame}[c]{Containers (e.g. docker)}

  \bbi
\item[] Why are they cool?
  \bi
  \item Capture your entire environment, so your project is \emph{for
    sure\/} fully reproducible.
  \ei
\item[] Why didn't we cover them?
    \bi
    \item A bit technical
    \ei
\item[] Where would we point you?
    \bi
  \item
    \href{https://ropensci.org/blog/blog/2014/10/23/introducing-rocker}{Rocker:
      Docker for R}
  \item \href{http://ropenscilabs.github.io/r-docker-tutorial/}{R
    Docker tutorial}
   \ei
\ei

 \note{
   Docker containers allow you to encapsulate your entire environment
   (including the operating system and all libraries) for
   \emph{full\/} reproducibility.
 }

\end{frame}











\begin{frame}[c]{R Markdown templates}

  \bbi
\item[] Why are they cool?
  \bi
  \item More complete control over the appearance of your document
  \ei
\item[] Why didn't we cover them?
    \bi
    \item A bit technical
    \ei
\item[] Where would we point you?
    \bi
  \item
    \href{http://rmarkdown.rstudio.com/developer_document_templates.html}{R
        Markdown documentation}
   \ei
\ei

 \note{
   Document templates give you full control over the appearance of
   your R Markdown document output.
 }

\end{frame}



\begin{frame}[c]{knitr Bootstrap}

  \bbi
\item[] Why is it cool?
  \bi
\item Allows for generation of slicker reports
  \ei
\item[] Why didn't we cover it?
    \bi
    \item A bit technical
    \ei
\item[] Where would we point you?
    \bi
  \item \href{https://github.com/jimhester/knitrBootstrap}{\tt github.com/jimhester/knitrBootstrap}
   \ei
\ei

 \note{
    bootstrap-styled reports
 }

\end{frame}



\begin{frame}[c]{GitHub pages}

  \bbi
\item[] Why are they cool?
  \bi
\item Webpages built entirely in Markdown, providing
   nicer interfaces to your content
  \ei
\item[] Why didn't we cover them?
    \bi
    \item Tangential to \emph{reproducible research}?
    \ei
\item[] Where would we point you?
    \bi
  \item \href{https://pages.github.com}{\tt pages.github.com}
  \item \href{http://kbroman.org/simple_site}{\tt kbroman.org/simple\_site}
  \item \href{https://bookdown.org/yihui/blogdown/}{\tt bookdown.org/yihui/blogdown}
   \ei
\ei

 \note{
    Slick web pages from a github repository
 }

\end{frame}




\begin{frame}[c]{Bookdown}

  \bbi
\item[] Why is it cool?
  \bi
\item Write a book (or book-like object) entirely in R Markdown
  \ei
\item[] Why didn't we cover it?
    \bi
    \item Not enough time
    \ei
\item[] Where would we point you?
    \bi
  \item \href{https://bookdown.org/yihui/bookdown/}{\tt bookdown.org/yihui/bookdown}
   \ei
\ei

 \note{
    Surprisingly easy way to create an e-book or website.
 }

\end{frame}







\begin{frame}[c]{workflowr}

  \bbi
\item[] Why is it cool?
  \bi
\item R package to help generate a website with time-stamped,
  versioned reports of analyses.
  \ei
\item[] Why didn't we cover it?
    \bi
    \item Not enough time
    \ei
\item[] Where would we point you?
    \bi
  \item \href{https://jdblischak.github.io/workflowr/}{\tt jdblischak.github.io/workflowr}
   \ei
\ei

 \note{
    Project template + R Markdown documents + simple commands to
    compile and share them on the web, via GitHub.
 }

\end{frame}





\begin{frame}[c]{Xaringan}

  \bbi
\item[] Why is it cool?
  \bi
\item Use R Markdown to make slides for a talk
  \ei
\item[] Why didn't we cover it?
    \bi
    \item Not enough time
    \ei
\item[] Where would we point you?
    \bi
  \item \href{http://github.com/yihui/xaringan}{\tt github.com/yihui/xaringan}
   \ei
\ei

 \note{
    You can make slides with R Markdown, too. xaringan is the
    recommended tool.
 }

\end{frame}





\begin{frame}[c]{Shiny!}

  \bbi
\item[] Why is it cool?
  \bi
\item Interactive pictures have pizzazz.
  \ei
\item[] Why didn't we cover it?
    \bi
    \item Tangential to \emph{reproducible research}?
    \ei
\item[] Where would we point you?
    \bi
  \item \href{http://shiny.rstudio.com/}{\tt shiny.rstudio.com}
  \item \href{http://shiny.rstudio.com/tutorial}{\tt shiny.rstudio.com/tutorial}
   \ei
\ei

 \note{
    Interactive R-based web apps
 }

\end{frame}










\begin{frame}[c]{Feedback we'd like from you (1)}

  {\lolit What motivated us to teach this course? \\
  What would we see as a positive outcome? \\[18pt]}

  \bi
\item Given this motivation, are we doing things right?
\item What motivated you to take this course?
\item Were there specific sessions you found really useful/really
    useless?
  \item Points you'd like us to expand on?
  \item Were there points you were hoping we'd cover that we didn't?
    \ei

  \note{}


\end{frame}



\begin{frame}[c]{Feedback we'd like from you (2)}

  \bi
\item Do you have examples/anecdotes you think we might be able to use
  that you'd be willing to share?
\item Were there ways we could've used time more effectively?
  \item Can you see things you learned in this course changing how you
    do things day to day?
    \bi
  \item Why or why not?
  \item Can we ask you again in 6 months?
  \item Can we ask you again in a year?
    \ei

 \item Could you write this down now? (anonymous is fine)
    \ei

  \note{}


\end{frame}




\end{document}
