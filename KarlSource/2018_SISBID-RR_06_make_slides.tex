\documentclass[12pt,t]{beamer}
\usepackage{graphicx}
\setbeameroption{hide notes}
\setbeamertemplate{note page}[plain]
\usepackage{listings}

% get rid of junk
\usetheme{default}
\beamertemplatenavigationsymbolsempty
\hypersetup{pdfpagemode=UseNone} % don't show bookmarks on initial view


% font
\usepackage{fontspec}
\setsansfont
  [ ExternalLocation = fonts/ ,
    UprightFont = *-regular ,
    BoldFont = *-bold ,
    ItalicFont = *-italic ,
    BoldItalicFont = *-bolditalic ]{texgyreheros}
\setbeamerfont{note page}{family*=pplx,size=\footnotesize} % Palatino for notes
% "TeX Gyre Heros can be used as a replacement for Helvetica"
% I've placed them in fonts/; alternatively you can install them
% permanently on your system as follows:
%     Download http://www.gust.org.pl/projects/e-foundry/tex-gyre/heros/qhv2.004otf.zip
%     In Unix, unzip it into ~/.fonts
%     In Mac, unzip it, double-click the .otf files, and install using "FontBook"

% named colors
\definecolor{offwhite}{RGB}{255,250,240}
\definecolor{gray}{RGB}{155,155,155}

\ifx\notescolors\undefined % slides
  \definecolor{foreground}{RGB}{255,255,255}
  \definecolor{background}{RGB}{24,24,24}
  \definecolor{title}{RGB}{107,174,214}
  \definecolor{subtitle}{RGB}{102,255,204}
  \definecolor{hilit}{RGB}{102,255,204}
  \definecolor{vhilit}{RGB}{255,111,207}
  \definecolor{lolit}{RGB}{155,155,155}
  \definecolor{codehilit}{RGB}{255,111,207}
\else % notes
  \definecolor{background}{RGB}{255,255,255}
  \definecolor{foreground}{RGB}{24,24,24}
  \definecolor{title}{RGB}{27,94,134}
  \definecolor{subtitle}{RGB}{22,175,124}
  \definecolor{hilit}{RGB}{122,0,128}
  \definecolor{vhilit}{RGB}{255,0,128}
  \definecolor{lolit}{RGB}{95,95,95}
  \definecolor{codehilit}{RGB}{24,24,24}
\fi
\definecolor{nhilit}{RGB}{128,0,128}  % hilit color in notes
\definecolor{nvhilit}{RGB}{255,0,128} % vhilit for notes

\newcommand{\hilit}{\color{hilit}}
\newcommand{\vhilit}{\color{vhilit}}
\newcommand{\nhilit}{\color{nhilit}}
\newcommand{\nvhilit}{\color{nvhilit}}
\newcommand{\lolit}{\color{lolit}}

% use those colors
\setbeamercolor{titlelike}{fg=title}
\setbeamercolor{subtitle}{fg=subtitle}
\setbeamercolor{institute}{fg=lolit}
\setbeamercolor{normal text}{fg=foreground,bg=background}
\setbeamercolor{item}{fg=foreground} % color of bullets
\setbeamercolor{subitem}{fg=lolit}
\setbeamercolor{itemize/enumerate subbody}{fg=lolit}
\setbeamertemplate{itemize subitem}{{\textendash}}
\setbeamerfont{itemize/enumerate subbody}{size=\footnotesize}
\setbeamerfont{itemize/enumerate subitem}{size=\footnotesize}

% page number
\setbeamertemplate{footline}{%
    \raisebox{5pt}{\makebox[\paperwidth]{\hfill\makebox[20pt]{\lolit
          \scriptsize\insertframenumber}}}\hspace*{5pt}}

% add a bit of space at the top of the notes page
\addtobeamertemplate{note page}{\setlength{\parskip}{12pt}}

% default link color
\hypersetup{colorlinks, urlcolor={hilit}}

\ifx\notescolors\undefined % slides
  % set up listing environment
  \lstset{language=bash,
          basicstyle=\ttfamily\scriptsize,
          frame=single,
          commentstyle=,
          backgroundcolor=\color{darkgray},
          showspaces=false,
          showstringspaces=false
          }
\else % notes
  \lstset{language=bash,
          basicstyle=\ttfamily\scriptsize,
          frame=single,
          commentstyle=,
          backgroundcolor=\color{offwhite},
          showspaces=false,
          showstringspaces=false
          }
\fi

% a few macros
\newcommand{\bi}{\begin{itemize}}
\newcommand{\bbi}{\vspace{24pt} \begin{itemize} \itemsep8pt}
\newcommand{\ei}{\end{itemize}}
\newcommand{\ig}{\includegraphics}
\newcommand{\subt}[1]{{\footnotesize \color{subtitle} {#1}}}
\newcommand{\ttsm}{\tt \small}
\newcommand{\ttfn}{\tt \footnotesize}
\newcommand{\figh}[2]{\centerline{\includegraphics[height=#2\textheight]{#1}}}
\newcommand{\figw}[2]{\centerline{\includegraphics[width=#2\textwidth]{#1}}}


%%%%%%%%%%%%%%%%%%%%%%%%%%%%%%%%%%%%%%%%%%%%%%%%%%%%%%%%%%%%%%%%%%%%%%
% end of header
%%%%%%%%%%%%%%%%%%%%%%%%%%%%%%%%%%%%%%%%%%%%%%%%%%%%%%%%%%%%%%%%%%%%%%

% title info
\title{6: GNU Make}
\author{}
\date{\href{https://bit.ly/2018rr}{\tt \color{foreground} bit.ly/2018rr}}

\begin{document}

% title slide
{
\setbeamertemplate{footline}{} % no page number here
\frame{
  \titlepage
  \note{Here, we'll talk about automating the entire workflow for a
    project using the tool GNU Make.
} } }


\begin{frame}{GNU Make}

  \bi
\item Automation the full project
\item Document dependencies
\item Only re-run things that need to be re-run
  \ei

  \note{
    GNU Make is an old tool that was originally for automating the
    compilation of large, complex programs. But it's useful much more
    generally, though it does have some quirks.

    In addition to automating, you'll be documenting the dependencies
    among the steps, and since the dependencies are defined, only
    stuff that needs to be re-run will be.
  }
\end{frame}



\begin{frame}[fragile,c]{Automate the process (GNU Make)}

\begin{center}
\begin{minipage}[c]{10.8cm}
\begin{semiverbatim}
\begin{lstlisting}[escapechar=!,linewidth=10.8cm]
!{\color{foreground}{R/analysis.html}}!: !{\color{foreground}{R/analysis.Rmd Data/cleandata.csv}}!
!{\color{foreground}{    cd R;R -e "rmarkdown::render('analysis.Rmd')"}}!

Data/cleandata.csv: R/prepData.R RawData/rawdata.csv
    cd R;R CMD BATCH prepData.R

RawData/rawdata.csv: Python/xls2csv.py RawData/rawdata.xls
    Python/xls2csv.py RawData/rawdata.xls > RawData/rawdata.csv
\end{lstlisting}
\end{semiverbatim}
\end{minipage}
\end{center}

\note{GNU Make is an old (and rather quirky) tool for automating the
  process of building computer programs. But it's useful much more
  broadly, and I find it valuable for automating the full process of
  data file manipulation, data cleaning, and analysis.

  In addition to {\nhilit automating} a complex process, it also
  {\nhilit documents} the process, including the dependencies among
  data files and scripts.
}
\end{frame}


\begin{frame}<handout:0>[fragile,c]{Automate the process (GNU Make)}

\addtocounter{framenumber}{-1}

\begin{center}
\begin{minipage}[c]{10.8cm}
\begin{semiverbatim}
\begin{lstlisting}[escapechar=!,linewidth=10.8cm]
!{\color{codehilit}{R/analysis.html}}!: !{\color{foreground}{R/analysis.Rmd Data/cleandata.csv}}!
!{\color{foreground}{    cd R;R -e "rmarkdown::render('analysis.Rmd')"}}!

Data/cleandata.csv: R/prepData.R RawData/rawdata.csv
    cd R;R CMD BATCH prepData.R

RawData/rawdata.csv: Python/xls2csv.py RawData/rawdata.xls
    Python/xls2csv.py RawData/rawdata.xls > RawData/rawdata.csv
\end{lstlisting}
\end{semiverbatim}
\end{minipage}
\end{center}

\end{frame}



\begin{frame}<handout:0>[fragile,c]{Automate the process (GNU Make)}

\addtocounter{framenumber}{-1}

\begin{center}
\begin{minipage}[c]{10.8cm}
\begin{semiverbatim}
\begin{lstlisting}[escapechar=!,linewidth=10.8cm]
!{\color{foreground}{R/analysis.html}}!: !{\color{codehilit}{R/analysis.Rmd Data/cleandata.csv}}!
!{\color{foreground}{    cd R;R -e "rmarkdown::render('analysis.Rmd')"}}!

Data/cleandata.csv: R/prepData.R RawData/rawdata.csv
    cd R;R CMD BATCH prepData.R

RawData/rawdata.csv: Python/xls2csv.py RawData/rawdata.xls
    Python/xls2csv.py RawData/rawdata.xls > RawData/rawdata.csv
\end{lstlisting}
\end{semiverbatim}
\end{minipage}
\end{center}
\end{frame}



\begin{frame}<handout:0>[fragile,c]{Automate the process (GNU Make)}

\addtocounter{framenumber}{-1}

\begin{center}
\begin{minipage}[c]{10.8cm}
\begin{semiverbatim}
\begin{lstlisting}[escapechar=!,linewidth=10.8cm]
!{\color{foreground}{R/analysis.html}}!: !{\color{foreground}{R/analysis.Rmd Data/cleandata.csv}}!
!{\color{codehilit}{    cd R;R -e "rmarkdown::render('analysis.Rmd')"}}!

Data/cleandata.csv: R/prepData.R RawData/rawdata.csv
    cd R;R CMD BATCH prepData.R

RawData/rawdata.csv: Python/xls2csv.py RawData/rawdata.xls
    Python/xls2csv.py RawData/rawdata.xls > RawData/rawdata.csv
\end{lstlisting}
\end{semiverbatim}
\end{minipage}
\end{center}
\end{frame}



\begin{frame}{Automation with GNU Make}

\vspace{24pt}

\bi
\itemsep12pt
\item {\tt Make} is for more than just compiling software
\item The {\color{hilit} essence} of what we're trying to do
\item Automates a workflow
\item Documents the workflow
\item Documents the dependencies among data files, code
\item Re-runs only the necessary code, based on what has changed
\ei


\note{People usually think of Make as a tool for automating the
  compilation of software, but it can be used much more generally.

  To me, Make is the essential tool for reproducible research:
  automation plus the documentation of dependencies and workflows.
}
\end{frame}





\begin{frame}[fragile]{Fancier example}

\begin{semiverbatim}
\begin{lstlisting}
FIG_DIR = Figs

mypaper.pdf: mypaper.tex $(FIG_DIR)/fig1.pdf $(FIG_DIR)/fig2.pdf
    pdflatex mypaper

# One line for both figures
$(FIG_DIR)/%.pdf: R/%.R
    cd R;R CMD BATCH $(<F)

# Use "make clean" to remove the PDFs
clean:
    rm *.pdf Figs/*.pdf
\end{lstlisting}
\end{semiverbatim}

\note{As I said, you can get really fancy with GNU Make.

  Use variables for directory names or compiler flags. (This example
  is not a good one.)

  Use pattern rules and automatic variables to avoid repeating
  yourself. With {\tt \%}, we have one line covering both
  {\tt fig1.pdf} and {\tt fig2.pdf}. The {\tt \$(<F)}
  is the file part of
  the first dependency. More on this later.

  Look at the manual for Make and the many online tutorials, such as
  the one from Software Carpentry, or {\tt http://kbroman.org/minimal\_make}.
}
\end{frame}



\begin{frame}[c]{Installing Make}

  \bbi
  \item On Macs, Make should be installed. Type \\
    ``{\tt make --version}''
    to check.

  \item On Windows, probably the easiest is to install
    \href{https://cran.r-project.org/bin/windows/Rtools/}{Rtools},
    which includes Make.
    \bi
    \item[]
      \href{https://cran.r-project.org/bin/windows/Rtools}{\tt
        \footnotesize cran.r-project.org/bin/windows/Rtools}
      \ei
      \ei

      \note{
        Installation of these sorts of command-line tools on Windows
        can be a bit difficult.
      }

\end{frame}


\begin{frame}[fragile]{How do you use Make?}

\vspace{6pt}

{\small
\bi
\item If you name your make file {\tt Makefile}, then just go into the
directory containing that file and type {\tt \color{hilit} make}

\item If you name your make file {\tt something.else}, then type \\
{\tt \color{hilit} make -f something.else}

\item Actually, the commands above will build the {\color{vhilit} first}
  target listed in the make file. So I'll often include something like
  the following.

\begin{quote}
{\tt \color{hilit} all: target1 target2 target3}
\end{quote}

  Then typing {\tt \color{hilit} make all} (or just {\tt
    \color{hilit} make}, if {\tt \color{hilit} all} is listed
  first in the file) will build all of those
  things.

\item To be build a specific target, type {\tt \color{hilit} make target}.
  For example, {\tt \color{hilit} make Figs/fig1.pdf}
\ei
}

\note{
  Details on the use.
}
\end{frame}





\begin{frame}[fragile]{Variables}

  \bbi
\item Define a {\color{vhilit} variable} like

  {\color{hilit} \verb|R_OPTS=--vanilla|}

\item Use it with a \verb|$| and \verb|()| or \verb|{}|, for example:

  {\color{hilit} \verb|R CMD BATCH $(R_OPTS) fig1.R|}

  \ei

\note{Variables are useful shorthand, for bits of code that you want to
  use repeatedly.}


\end{frame}





\begin{frame}[fragile]{Automatic variables}

There are a bunch of {\color{vhilit} automatic variables} that you can use to save
yourself a lot of typing.

\bigskip

Here are the ones I use most:

\bigskip

{\small
\def\arraystretch{1.5}
\begin{tabular}{l@{\hspace{1cm}}l}
{\color{hilit} \verb|$@|} & the file name of the target \\
{\color{hilit} \verb|$<|} &  the name of the first dependency \\
{\color{hilit} \verb|$^|} &  the names of all dependencys \\
{\color{hilit} \verb|$(@D)|} &  the directory part of the target \\
{\color{hilit} \verb|$(@F)|} &  the file part of the target \\
{\color{hilit} \verb|$(<D)|} &  the directory part of the first dependency \\
{\color{hilit} \verb|$(<F)|} &  the file part of the first dependency
\end{tabular}
}

\note{You'll see on the next slide how automatic variables get used.}


\end{frame}






\end{document}
