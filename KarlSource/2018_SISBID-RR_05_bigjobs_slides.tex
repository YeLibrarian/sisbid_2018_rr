\documentclass[12pt,t]{beamer}
\usepackage{graphicx}
\setbeameroption{hide notes}
\setbeamertemplate{note page}[plain]
\usepackage{listings}

% get rid of junk
\usetheme{default}
\beamertemplatenavigationsymbolsempty
\hypersetup{pdfpagemode=UseNone} % don't show bookmarks on initial view


% font
\usepackage{fontspec}
\setsansfont
  [ ExternalLocation = fonts/ ,
    UprightFont = *-regular ,
    BoldFont = *-bold ,
    ItalicFont = *-italic ,
    BoldItalicFont = *-bolditalic ]{texgyreheros}
\setbeamerfont{note page}{family*=pplx,size=\footnotesize} % Palatino for notes
% "TeX Gyre Heros can be used as a replacement for Helvetica"
% I've placed them in fonts/; alternatively you can install them
% permanently on your system as follows:
%     Download http://www.gust.org.pl/projects/e-foundry/tex-gyre/heros/qhv2.004otf.zip
%     In Unix, unzip it into ~/.fonts
%     In Mac, unzip it, double-click the .otf files, and install using "FontBook"

% named colors
\definecolor{offwhite}{RGB}{255,250,240}
\definecolor{gray}{RGB}{155,155,155}

\ifx\notescolors\undefined % slides
  \definecolor{foreground}{RGB}{255,255,255}
  \definecolor{background}{RGB}{24,24,24}
  \definecolor{title}{RGB}{107,174,214}
  \definecolor{subtitle}{RGB}{102,255,204}
  \definecolor{hilit}{RGB}{102,255,204}
  \definecolor{vhilit}{RGB}{255,111,207}
  \definecolor{lolit}{RGB}{155,155,155}
  \definecolor{codehilit}{RGB}{255,111,207}
\else % notes
  \definecolor{background}{RGB}{255,255,255}
  \definecolor{foreground}{RGB}{24,24,24}
  \definecolor{title}{RGB}{27,94,134}
  \definecolor{subtitle}{RGB}{22,175,124}
  \definecolor{hilit}{RGB}{122,0,128}
  \definecolor{vhilit}{RGB}{255,0,128}
  \definecolor{lolit}{RGB}{95,95,95}
  \definecolor{codehilit}{RGB}{24,24,24}
\fi
\definecolor{nhilit}{RGB}{128,0,128}  % hilit color in notes
\definecolor{nvhilit}{RGB}{255,0,128} % vhilit for notes

\newcommand{\hilit}{\color{hilit}}
\newcommand{\vhilit}{\color{vhilit}}
\newcommand{\nhilit}{\color{nhilit}}
\newcommand{\nvhilit}{\color{nvhilit}}
\newcommand{\lolit}{\color{lolit}}

% use those colors
\setbeamercolor{titlelike}{fg=title}
\setbeamercolor{subtitle}{fg=subtitle}
\setbeamercolor{institute}{fg=lolit}
\setbeamercolor{normal text}{fg=foreground,bg=background}
\setbeamercolor{item}{fg=foreground} % color of bullets
\setbeamercolor{subitem}{fg=lolit}
\setbeamercolor{itemize/enumerate subbody}{fg=lolit}
\setbeamertemplate{itemize subitem}{{\textendash}}
\setbeamerfont{itemize/enumerate subbody}{size=\footnotesize}
\setbeamerfont{itemize/enumerate subitem}{size=\footnotesize}

% page number
\setbeamertemplate{footline}{%
    \raisebox{5pt}{\makebox[\paperwidth]{\hfill\makebox[20pt]{\lolit
          \scriptsize\insertframenumber}}}\hspace*{5pt}}

% add a bit of space at the top of the notes page
\addtobeamertemplate{note page}{\setlength{\parskip}{12pt}}

% default link color
\hypersetup{colorlinks, urlcolor={hilit}}

\ifx\notescolors\undefined % slides
  % set up listing environment
  \lstset{language=bash,
          basicstyle=\ttfamily\scriptsize,
          frame=single,
          commentstyle=,
          backgroundcolor=\color{darkgray},
          showspaces=false,
          showstringspaces=false
          }
\else % notes
  \lstset{language=bash,
          basicstyle=\ttfamily\scriptsize,
          frame=single,
          commentstyle=,
          backgroundcolor=\color{offwhite},
          showspaces=false,
          showstringspaces=false
          }
\fi

% a few macros
\newcommand{\bi}{\begin{itemize}}
\newcommand{\bbi}{\vspace{24pt} \begin{itemize} \itemsep8pt}
\newcommand{\ei}{\end{itemize}}
\newcommand{\ig}{\includegraphics}
\newcommand{\subt}[1]{{\footnotesize \color{subtitle} {#1}}}
\newcommand{\ttsm}{\tt \small}
\newcommand{\ttfn}{\tt \footnotesize}
\newcommand{\figh}[2]{\centerline{\includegraphics[height=#2\textheight]{#1}}}
\newcommand{\figw}[2]{\centerline{\includegraphics[width=#2\textwidth]{#1}}}


%%%%%%%%%%%%%%%%%%%%%%%%%%%%%%%%%%%%%%%%%%%%%%%%%%%%%%%%%%%%%%%%%%%%%%
% end of header
%%%%%%%%%%%%%%%%%%%%%%%%%%%%%%%%%%%%%%%%%%%%%%%%%%%%%%%%%%%%%%%%%%%%%%

% title info
\title{5: EDA and Big Jobs}
\author{}
\date{\href{https://bit.ly/2018rr}{\tt \color{foreground} bit.ly/2018rr}}

\begin{document}

% title slide
{
\setbeamertemplate{footline}{} % no page number here
\frame{
  \titlepage
  \note{Here, we'll talk about capturing exploratory data analysis
    (EDA), handling big data, and managing jobs.

  Regarding exploratory data analysis, we want to capture the whole
  process: what you're trying to do, what you're thinking about, what
  you're seeing, and what you're concluding and why. And we want to do
  so without getting in the way of the creative process.

  Regarding long-running jobs, the problems are: (a) it's hard for
  someone to re-do all of that work, and (b) large-scale calculations
  tend to be organized in a system-dependent way, so even if time
  weren't a factor, it'd be harder to transfer the calculations to
  another system.
} } }



\begin{frame}{Exploratory data analysis}

\vspace{24pt}

\bi
\item what were you trying to do?
\item what were you thinking about?
\item what did you observe?
\item what did you conclude, and why?
\ei

\note{We want to be able to capture the full outcome of exploratory
  data analysis.

  But we don't want to inhibit the creative flow. How to capture this
  stuff?
}
\end{frame}


\begin{frame}{Avoid}

\vspace{24pt}

\bi
\item "How did I create this plot?"
\item "Why did I decide to omit those six samples?"
\item "Where (on the web) did I find these data?"
\item "What was that interesting gene?"
\ei

\note{I've said all of these things to myself.
}
\end{frame}



\begin{frame}{Basic principles}

\vspace{24pt}

\bi
\item[] {\hilit Step 1}: slow down and document.
\item[] {\hilit Step 2}: have sympathy for your future self.
\item[] {\hilit Step 3}: have a system.
\ei

\note{I can't emphasize these things enough.

  If you're not {\nhilit thinking} about keeping track of things, you
  won't keep track of things.

  One thing I like to do: write a set of comments describing my basic
  plan, and then fill in the code afterwards. It forces you to think
  things through, and then you'll have at least a rough sense of what
  you were doing, even if you don't take the time to write further
  comments.
}
\end{frame}

\begin{frame}{Capturing EDA}


\bbi
\item copy-and-paste from an R file
\item grab code from the {\tt .Rhistory} file
\item Write an informal R Markdown file; \\
  make use of the ``R Notebook'' features.
\item Write code for use with the KnitR function {\tt spin()}
\bi
\item[] Comments like \; {\hilit \tt \#' This will become text}
\item[] Chunk options like so: \; {\hilit \tt \#+ chunk\_label, echo=FALSE}
\ei
\ei

\note{There are a number of techniques you can use to capture the EDA
  process.

  You don't need to save all of the figures, but you do need to save
  the code and write down your motivation, observations, and
  conclusions.

  I usually start out with a plain R file and then move to more formal
  R Markdown or AsciiDoc reports.
}
\end{frame}


\begin{frame}[fragile]{A file to {\tt spin()}}

\vspace{24pt}

\begin{lstlisting}
#' This is a simple example of an R file for use with spin().

#' We'll start by setting the seed for the RNG.
set.seed(53079239)

#' We'll first simulate some data with x ~ N(mu=10, sig=5) and
#' y = 2x + e, where e ~ N(mu=0, sig=2)
x <- rnorm(100, 10, 5)
y <- 2*x + rnorm(100, 0, 2)

#' Here's a scatterplot of the data.
plot(x, y, pch=21, bg="slateblue", las=1)
\end{lstlisting}

\note{Here's an example R file for use with {\tt spin()}.
}
\end{frame}



\begin{frame}[c]{Activity}

Try out {\tt knitr::spin()}:

  \bi
      \item Create an R script using
      \bi
          \item {\tt \#'} for text
          \item {\tt \#+} for chunk options
      \ei
      \item Compile the script to HTML using {\tt knitr::spin()}
      \item Open the resulting HTML file.
  \ei

  \note{
    A brief activity to try out {\tt knitr::spin()}.
    }

\end{frame}




\begin{frame}{Big jobs}

\bbi
\item You don't want {\tt knitr} running for a year.

\item You don't want to re-run things if you don't have to.
\ei

\note{
  It may not seem like ``big jobs'' are that big of a deal, but in my
  mind this is the only real difficulty in reproducible research.

  Okay, there's also the difficulty that some data can't be generally
  distributed due to confidentiality issues.

  But aside from subjects' confidentiality, the main problem is
  how to manage and capture the really long-running computational
  analyses where even on the same system it can be a gargantuan effort
  to reproduce the results.
}
\end{frame}





\begin{frame}{Biggish jobs in knitr}

\bbi
\item Manual caching
\item Built-in {\tt cache=TRUE}
\item Split the work into a separate subdirectory.
\ei

\note{
  You can put big computations within an R Markdown file, but
  {\nhilit personally} I don't want to wait more than a couple of
  minutes for it to compile. If it's going to take longer than that,
  I'll split things up.

  And if you are going to have some large computations with knitr, you
  won't want to re-run {\nhilit all} of them every time you make even
  the smallest change to the text!

  That's where you want to {\nhilit cache} some computations: save the
  results and just load the results rather than re-run the
  code. Unless the {\nhilit code} changes, in which case you {\nhilit
  do} want to re-run it (and any other code that may depend on the
  results).
}
\end{frame}


\begin{frame}[c,fragile]{Manual caching}

\begin{lstlisting}
```{r a_code_chunk}
file <- "cache/myfile.RData"

if(file.exists(file)) {
  load(file)
} else{

  ....

  save(object1, object2, object3, file=file)
}
```
\end{lstlisting}

\note{
  This is the ``by hand'' approach. If the file doesn't exist, run the
  relevant code and save the needed results to the file. If the file
  does exist, just load the file and skip the code.

  If you want (or need) to re-run the code, you need to delete the
  file {\nhilit manually}.

  One issue: if you want the code to actually be {\nhilit shown}, you
  need to repeat the code: in a chunk that is shown but isn't run, and
  then in this chunk that is run but isn't shown.

  You need to be very careful about dependencies.
}
\end{frame}


\begin{frame}[c,fragile]{Chunk references}

\begin{lstlisting}
```{r not_run, eval=FALSE}
awesome_code_here <- 0
```

```{r not_shown, echo=FALSE}
file <- "cache/myfile.RData"

if(file.exists(file)) {
  load(file)
} else{
<<not_run>>
  save(awesome_code_here, file=file)
}
```
\end{lstlisting}

\note{
  Here's how I'd avoid repeated code: use chunk references.

  The {\tt <<not\_run>>} is replaced by the code from the chunk with
  that label.
}
\end{frame}



\begin{frame}[c]{A cache gone bad}

\figh{Figs/cache_gone_bad.png}{0.65}

\note{
  This is the sort of thing that can happen with manual caching.

  This is Fig.\ 11.14 from my book, A guide to QTL mapping with R/qtl.

  I saw it immediately upon flipping through my first paper copy of
  the printed book.

  I'd cached some results, but then changed the underlying software
  in a fundamental way and didn't update the cache.
}
\end{frame}



\begin{frame}[fragile]{Knitr's cache system}

\vspace{18pt}

\begin{lstlisting}
```{r chunk_name, cache=TRUE}
load("a_big_file.RData")
med <- apply(object, 2, median, na.rm=TRUE)
```
\end{lstlisting}

\bbi
\item Chunk is re-run if edited.
\item Otherwise, objects from previous run are loaded.
\item Don't cache things with side effects
  \bi
  \item[] e.g., {\tt options()}, {\tt par()}
  \ei
\ei


\note{
  Knitr has a nice built-in system for caching.

  A chunk with {\tt cache=TRUE} will be run once and then all objects
  saved. In future runs, the code won't be run, but rather the cached
  objects will be loaded.

  But some things {\nvhilit shouldn't} be cached. ``Side effects''
  change the state of things; mostly, this is changing global
  variables. If these are placed in a cached chunk, the side effects
  won't be captured when the cache is loaded.
}
\end{frame}




\begin{frame}[fragile]{Cache dependencies}

\vspace{18pt}

Manual dependencies
\bigskip

\begin{lstlisting}
```{r chunkA, cache=TRUE}
Sys.sleep(2)
x <- 5
```

```{r chunkB, cache=TRUE, dependson="chunkA"}
Sys.sleep(2)
y <- x + 1
```

```{r chunkC, cache=TRUE, dependson="chunkB"}
Sys.sleep(2)
z <- y + 1
```
\end{lstlisting}

\note{
  You can indicate dependencies among chunks. Here, if {\tt chunkA} is
  re-run, the other two will be as well.
}
\end{frame}


\begin{frame}[fragile]{Cache dependencies}

\vspace{18pt}

Automatic dependencies
\bigskip

\begin{lstlisting}
```{r setup, include=FALSE}
opts_chunk$set(autodep = TRUE)
dep_auto()
```
\end{lstlisting}


\note{
  There's also an automatic system for determining dependencies among chunks.

  I've not used it.
}
\end{frame}





\begin{frame}{Parallel computing}

\vspace{18pt}

If your computer has multiple processors, use {\tt library(parallel)}
to make use of them.

\bbi
\item {\tt detectCores()}
\item {\tt \hilit RNGkind("L'Ecuyer-CMRG")} and {\tt \hilit mclapply} (Unix/Mac)
\item {\tt \hilit makeCluster}, {\tt \hilit clustersetRNGStream}, {\tt
  \hilit clusterApply}, and {\tt \hilit stopCluster} (Windows)
\ei

\note{
  R has a built-in package for performing parallel computations. A
  number of instances of R are invoked, calculations begun, and then
  the results brought back together.

  The code can be a bit ugly, but it's not so bad once you get used to
  it.

  See the links on the resources page for Karl's Tools for
  Reproducible Research course,
  {\tt http://kbroman.github.io/Tools4RR/pages/resources.html}

}
\end{frame}


\begin{frame}{Systems for distributed computing}

\bbi
\item At UW-Madison, we use \href{http://research.cs.wisc.edu/htcondor}{HTCondor}
\item There are
  \href{http://en.wikipedia.org/wiki/Comparison_of_cluster_software}{oddles
    of similar systems}
\item "By hand"
\ei

\note{
  For really big jobs, you'll want to distribute the computations
  across multiple computers.

  At UW-Madison, we use the locally-developed software HTCondor. This
  provides a way of distributing enormous numbers of jobs across a
  heterogeneous set of computers and collecting the results.

  There exists other, similar systems for distributing and managing
  jobs across clusters of computers.

  My own approach is more primitive: I have a Perl script that
  converts a template R script into a bunch of R input files (by
  replacing every instance of ``{\tt SUB}'' in the template with a
  job-specific index). It also creates a script to set those running.

  In either case, you'd write another R script to combine the results
  from the multiple jobs.
}
\end{frame}


\begin{frame}{Simulations}

\bbi
\item Computer simulations require RNG seeds ({\tt .Random.seed} in R).
\item Multiple parallel jobs need different seeds.
\item Don't rely on the current seed, or on having it generated from
  the clock.
\item Use something like {\hilit \tt set.seed(91820205 + i)}
\item An alternative is create a big batch of simulated data sets in
  advance.
\ei

\note{
  RNG = Random number generator

  Simulations split across multiple CPUs each need their own seed.
  In R, the seed is saved as {\tt .Random.seed}; if you start all of
  the simulations from the same directory, they could all get exactly
  the same seed.

  I tend to include a call to {\tt set.seed} at the top of each R
  script, with the seed being some big number plus an index for the
  job.

  You could, alternatively, generate all of the simulated data sets in
  advance. An advantage of this is that it'd be easier to reproduce
  the results later. Just be sure to save (and document) the code you
  used to generate the data.
}
\end{frame}



\begin{frame}{Save everything}

\bbi
\item RNG seeds
\item input
\item output
\item version numbers, with {\tt sessionInfo()}
\item raw results
\item script to combine results
\item combined results
\item {\tt ReadMe} describing the point
\ei

\note{
  This stuff (particularly code input \& text output) doesn't take up
  much space. Compartmentalize it and save it.
}
\end{frame}


\begin{frame}{Compartmentalize big jobs}

\bbi
\item Separate directory for each batch of big computations.
\item Collect the results in `.RData` or `.rds` files.
\item KnitR-based documents for the analysis/use of those results.
\ei

\note{
  My approach, for projects that involve big
  computations: compartmentalize those big computations into chunks,
  each in a separate directory to contain all of the materials and
  results.

  I use GNU Make to handle the combination of those results as
  well as the compilation of any KnitR-based files that describe and
  carry out the further analyses.

  This won't capture the entire workflow, but it will indicate
  almost all of it, and the big jobs will be encapsulated as
  subdirectories, and the source of the major results will be
  indicated.
}
\end{frame}


\begin{frame}{Potential problems}

\bbi
\item Forgetting {\tt save()} in your distributed jobs
\item A bug in the {\tt save()} command
\item Keeping things synchronized
  \bi
  \item Have you re-run the big jobs when upstream data were revised?
  \ei
\ei

\note{
  These are common mistakes I make.

  I forget the {\tt save} command and so run a ton of computations and
  then get no results.

  Or my {\tt save} command has an error and so I run a ton of
  computations and then it dies on the last line of the script.

  Or the upstream data changes and I fail to recognize that I need to
  re-run the big jobs that depend on it.
}
\end{frame}





\begin{frame}{Big jobs summary}

\bbi
\item Careful organization and modularization.
\item Save everything.
\item Document everything.
\item Learn the basic skills for distributed computing.
\ei

\note{
  Research with long-running computations are hard to make fully
  reproducible. Modularize the big jobs and document their
  purpose. And document the relationships among things.
}
\end{frame}



\begin{frame}[c]{Activity}

Try out the knitr cache system.

  \bi
      \item Create an RMarkdown document with multiple dependent
        chunks.
      \item Maybe add {\tt Sys.sleep(5)} to slow things down, as if
        the jobs were taking a while.
      \item Compile the document.
      \item Edit a chunk.
      \item Re-compile the document and see what was actually run and
        what was taken from cache.
  \ei

  \note{
    A brief activity to try out caching in RMarkdown
  }

\end{frame}




\end{document}
